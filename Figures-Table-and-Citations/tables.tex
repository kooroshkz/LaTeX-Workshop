\documentclass{article}
\usepackage{booktabs}  % For professional-looking tables
\usepackage{multirow}  % For multi-row cells
\usepackage{array}     % Customizing column formats
\usepackage[table,xcdraw]{xcolor}  % For table coloring

% Title Page
\title{LaTeX Workshop: Tables}
\author{
    \begin{tabular}{c c c}
        Dalia Kamalzadeh & \hspace{2cm} & Koorosh Komeili Zadeh \\
        Student Mentor & & Student Mentor \\
        Universiteit Leiden & & Universiteit Leiden
    \end{tabular}
}
\date{}
\begin{document}
\maketitle

\section*{Introduction to Tables}

In LaTeX, tables can vary from simple structures to complex ones that suit a variety of use cases, including professional reports and academic papers. Let's explore several types.

\subsection*{1. Simple Tables with Borders}

\textbf{Code:}
\begin{verbatim}
\begin{tabular}{|c|c|c|}
    \hline
    Name & Age & Score \\
    \hline
    Alice & 21 & 85 \\
    Bob & 22 & 90 \\
    Charlie & 20 & 78 \\
    \hline
\end{tabular}
\end{verbatim}

\textbf{Output:}
\begin{table}[h]
    \centering
    \begin{tabular}{|c|c|c|}
        \hline
        \textbf{Name} & \textbf{Age} & \textbf{Score} \\
        \hline
        Alice & 21 & 85 \\
        Bob & 22 & 90 \\
        Charlie & 20 & 78 \\
        \hline
    \end{tabular}
    \caption{Simple table with borders}
\end{table}

---

\subsection*{2. Professional Tables (Using \texttt{booktabs})}

You can create cleaner tables without vertical borders using the \texttt{booktabs} package.

\textbf{Code:}
\begin{verbatim}
\begin{tabular}{lcc}
    \toprule
    \textbf{Name} & \textbf{Age} & \textbf{Score} \\
    \midrule
    Alice & 21 & 85 \\
    Bob & 22 & 90 \\
    Charlie & 20 & 78 \\
    \bottomrule
\end{tabular}
\end{verbatim}

\textbf{Output:}
\begin{table}[h]
    \centering
    \begin{tabular}{lcc}
        \toprule
        \textbf{Name} & \textbf{Age} & \textbf{Score} \\
        \midrule
        Alice   & 21  & 85 \\
        Bob     & 22  & 90 \\
        Charlie & 20  & 78 \\
        \bottomrule
    \end{tabular}
    \caption{Professional table using \texttt{booktabs}}
\end{table}

---

\subsection*{3. Tables with Colors}

You can apply colors to rows or cells to highlight important data using the \texttt{xcolor} package.

\textbf{Code:}
\begin{verbatim}
\begin{tabular}{|c|c|c|}
    \hline
    \rowcolor{lightgray} \textbf{Name} & \textbf{Age} & \textbf{Score} \\
    \hline
    Alice   & 21  & 85 \\
    \rowcolor{yellow} Bob     & 22  & 90 \\
    Charlie & 20  & 78 \\
    \hline
\end{tabular}
\end{verbatim}

\textbf{Output:}
\begin{table}[h]
    \centering
    \begin{tabular}{|c|c|c|}
        \hline
        \rowcolor{lightgray} \textbf{Name} & \textbf{Age} & \textbf{Score} \\
        \hline
        Alice   & 21  & 85 \\
        \rowcolor{yellow} Bob     & 22  & 90 \\
        Charlie & 20  & 78 \\
        \hline
    \end{tabular}
    \caption{Table with row colors}
\end{table}

---

\subsection*{4. Multi-row and Multi-column Tables}

You can merge rows and columns to make complex tables using the \texttt{multirow} package.

\textbf{Code:}
\begin{verbatim}
\begin{tabular}{|c|c|c|}
    \hline
    \multirow{2}{*}{\textbf{Name}} & \multicolumn{2}{c|}{\textbf{Scores}} \\
    \cline{2-3}
    & Test 1 & Test 2 \\
    \hline
    Alice & 85 & 90 \\
    Bob   & 75 & 85 \\
    \hline
\end{tabular}
\end{verbatim}

\textbf{Output:}
\begin{table}[h]
    \centering
    \begin{tabular}{|c|c|c|}
        \hline
        \multirow{2}{*}{\textbf{Name}} & \multicolumn{2}{c|}{\textbf{Scores}} \\
        \cline{2-3}
                                       & \textbf{Test 1} & \textbf{Test 2} \\
        \hline
        Alice   & 85  & 90 \\
        Bob     & 75  & 85 \\
        \hline
    \end{tabular}
    \caption{Multi-row and multi-column table}
\end{table}

---

\section*{Final Thoughts on Tables}
Tables are a powerful tool in LaTeX, offering a wide range of options for presenting data clearly and effectively. By combining different packages and customization options, you can tailor tables to fit your document's needs.

\end{document}
