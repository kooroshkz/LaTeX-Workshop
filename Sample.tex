\documentclass{article}

% Packages for advanced math
\usepackage{amsmath}

% Begin the document
\begin{document}

% Title Page
\title{LaTeX Workshop}
\author{Dalia Kamal Zadeh and Koorosh Komeili Zadeh}
\date{}
\maketitle

\section*{LaTeX Basics}

\subsection*{Document Structure}
A basic LaTeX document starts with:
\begin{verbatim}
\documentclass{article}
\begin{document}
\end{document}
\end{verbatim}

\subsection*{Packages}
Packages add extra functionality. For example, to use advanced math commands:
\begin{verbatim}
\usepackage{amsmath}
\end{verbatim}

\subsection*{Comments}
Use the \% symbol to add comments in your LaTeX code:
\begin{verbatim}
% This is a comment in LaTeX.
\end{verbatim}

\newpage

\section*{Writing Math in LaTeX}

\subsection*{Inline Equations}
An inline equation example: \( a^2 + b^2 = c^2 \).

\subsection*{Displayed Equations}
Displayed equations are written in their own line, like this:
\[
E = mc^2
\]

\subsection*{Common Math Symbols}
Some common math symbols in LaTeX:
\begin{itemize}
    \item Fractions: \verb|\frac{a}{b}| produces \(\frac{a}{b}\)
    \item Summation: \verb|\sum| produces \(\sum\)
    \item Integral: \verb|\int| produces \(\int\)
    \item Greek Letters: \verb|\alpha| produces \(\alpha\), \verb|\pi| produces \(\pi\)
\end{itemize}

\subsection*{Examples for Calculus}
You can write calculus-related expressions using LaTeX:
\begin{enumerate}
    \item Derivatives: \(\frac{dy}{dx}\)
    \item Integrals: \(\int_0^1 x^2 \, dx\)
    \item Limits: \(\lim_{x \to 0} \frac{1}{x}\)
\end{enumerate}

% End the document
\end{document}