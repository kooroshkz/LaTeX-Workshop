\documentclass{report}
\usepackage{geometry}    % For page layout
\usepackage{fancyhdr}    % Custom headers and footers
\usepackage{hyperref}    % For hyperlinks and PDF navigation
\usepackage{lipsum}      % For placeholder text
\usepackage{graphicx}    % For images in title page
\usepackage[toc]{appendix} % For appendices
\usepackage{makeidx}     % For creating an index

\geometry{a4paper, margin=1in}

% Title Page
\title{LaTeX Workshop: Document Structure}
\author{
    \begin{tabular}{c c c}
        Dalia Kamalzadeh & \hspace{2cm} & Koorosh Komeili Zadeh \\
        Student Mentor & & Student Mentor \\
        Universiteit Leiden & & Universiteit Leiden
    \end{tabular}
}
\date{}
\makeindex % Create an index in the document

\begin{document}
\maketitle

% Adding Table of Contents
\tableofcontents
\newpage

\section*{Introduction to Document Structure}

LaTeX allows you to create complex documents with chapters, sections, and custom layouts. Here's how to make the most out of LaTeX's structure options.

\section{Basic Article Structure}
The most basic LaTeX document looks like this:
\begin{verbatim}
\documentclass{article}
\begin{document}
    \section{Introduction}
    Your content goes here...
\end{document}
\end{verbatim}

\section{Multi-chapter Reports}
For larger documents like reports or thesis, use the \texttt{report} class and add chapters.

\textbf{Code:}
\begin{verbatim}
\documentclass{report}
\begin{document}
    \chapter{Introduction}
    \section{Background}
\end{document}
\end{verbatim}

\textbf{Output:}
- \textbf{Chapter 1}: Introduction
  - \textbf{Section 1.1}: Background

---

\section{Customizing Page Layout}
You can customize margins, paper size, and layout using the \texttt{geometry} package.

\begin{verbatim}
\usepackage[a4paper, margin=1in]{geometry}
\end{verbatim}

---

\section{Custom Headers and Footers (Using \texttt{fancyhdr})}
LaTeX makes it easy to customize headers and footers with the \texttt{fancyhdr} package.

\textbf{Code:}
\begin{verbatim}
\usepackage{fancyhdr}
\pagestyle{fancy}
\fancyhf{}
\fancyhead[L]{LaTeX Workshop}
\fancyfoot[C]{Page \thepage}
\end{verbatim}

---

\section{Table of Contents}
To automatically generate a table of contents based on your chapters and sections, use the \texttt{\textbackslash tableofcontents} command:
\begin{verbatim}
\tableofcontents
\end{verbatim}
This will create a list of all chapters, sections, and subsections in your document.

---

\section{Cross-Referencing Sections and Figures}
You can reference sections, figures, and tables using the \texttt{label} and \texttt{ref} commands. This will automatically update numbers when they change.

\textbf{Code:}
\begin{verbatim}
\section{Introduction} \label{sec:intro}
In Section \ref{sec:intro}, we introduce the concept of...

\begin{figure}[h]
    \includegraphics{image.png}
    \caption{Sample Figure}
    \label{fig:sample}
\end{figure}

Refer to Figure \ref{fig:sample} for an example.
\end{verbatim}

---

\section{Appendices}
You can include appendices for additional material at the end of your document using the \texttt{appendix} package.

\textbf{Code:}
\begin{verbatim}
\begin{appendices}
    \chapter{Additional Figures}
    More details here...

    \chapter{Raw Data}
    Include raw data...
\end{appendices}
\end{verbatim}

---

\section{Customizing the Title Page}
You can customize the title page by adding logos, more detailed information, and adjusting formatting.

\textbf{Code:}
\begin{verbatim}
\title{
    \includegraphics[width=0.3\textwidth]{university-logo.png} \\[1cm]
    \textbf{LaTeX Workshop: Document Structure} \\
    \vspace{0.5cm}
    \large Universiteit Leiden
}
\author{
    \textbf{Dalia Kamalzadeh and Koorosh Komeili Zadeh} \\
    Student Mentors
}
\date{}
\end{verbatim}

This will add a university logo at the top of the title page and adjust the author and title styling.

---

\section{Creating an Index}
To create an index, use the \texttt{makeidx} package and the \texttt{\textbackslash index} command to mark terms in the text. LaTeX will then automatically generate an index based on those terms.

\textbf{Code:}
\begin{verbatim}
\usepackage{makeidx}
\makeindex

% In the document, use \index to add terms to the index
Here we introduce \texttt{LaTeX} \index{LaTeX}, a typesetting system.

% At the end of the document, generate the index
\printindex
\end{verbatim}

---

\section{Adding Hyperlinks and PDF Bookmarks}
To enable clickable links and navigation in your PDF, use the \texttt{hyperref} package. This makes section links clickable in the table of contents and creates a more interactive PDF.

\textbf{Code:}
\begin{verbatim}
\usepackage{hyperref}
\hypersetup{
    colorlinks=true,
    linkcolor=blue,
    filecolor=magenta,
    urlcolor=cyan,
    pdftitle={LaTeX Workshop},
    bookmarks=true
}
\end{verbatim}

This will color the links in your PDF and enable bookmarks in the PDF viewer. For example: \href{https://youtu.be/dQw4w9WgXcQ?si=_6pb5J2xQoMNWOeA}{Click this link}

---

\chapter{Conclusion}
In this guide, we explored:
\begin{itemize}
    \item The basic structure of a LaTeX document, including articles and reports.
    \item Custom page layouts, headers, and footers.
    \item Advanced features such as table of contents, cross-referencing, and appendices.
    \item Creating a custom title page and index.
    \item Adding hyperlinks and bookmarks to make your PDFs more interactive.
\end{itemize}

With these techniques, you can create highly structured, professional LaTeX documents for any purpose.

\newpage
% Print the index at the end of the document
\printindex

\end{document}
