\documentclass{article}
\usepackage{amsmath}
\usepackage{amssymb}
\usepackage{mathtools} % Enhanced math capabilities
\usepackage{bm}        % For bold math symbols

% Title Page
\title{LaTeX Workshop: Writing Math in LaTeX}
\author{
    \begin{tabular}{c c c}
        Dalia Kamalzadeh & \hspace{2cm} & Koorosh Komeili Zadeh \\
        Student Mentor & & Student Mentor \\
        Universiteit Leiden & & Universiteit Leiden
    \end{tabular}
}
\date{}
\begin{document}
\maketitle

\section*{Writing Math in LaTeX}

LaTeX is ideal for writing complex math equations, symbols, and formulas. Let's explore some advanced features and techniques.

---

\subsection*{1. Aligning Equations}

The \texttt{align} environment helps align multi-line equations:

\begin{align}
    x^2 + y^2 &= z^2 \\
    E &= mc^2 \\
    a + b &= c + d
\end{align}

You can also suppress numbering on specific lines using \texttt{\\nonumber}:

\begin{align}
    x^2 + y^2 &= z^2 \nonumber \\
    a + b &= c + d
\end{align}

---

\subsection*{2. Advanced Math Symbols with \texttt{amssymb}}

The \texttt{amssymb} package provides access to various advanced math symbols:

\[
\therefore, \because, \infty, \sum_{i=1}^{n}i, \Rightarrow, \implies
\]

You can also write set theory symbols like:

\[
A \cup B, \quad A \cap B, \quad \mathbb{R}, \quad \mathbb{N}
\]

---

\subsection*{3. Matrices and Vectors}

LaTeX can handle matrices and vectors using the \texttt{bmatrix} or \texttt{pmatrix} environments. Here's an example of a 2x2 matrix:

\[
\begin{bmatrix}
    a & b \\
    c & d
\end{bmatrix}
\]

And a vector:

\[
\mathbf{v} = \begin{pmatrix}
    v_1 \\
    v_2 \\
    v_3
\end{pmatrix}
\]

You can use other delimiters like parentheses or vertical lines for determinants:

\[
\det(A) = \begin{vmatrix}
    a & b \\
    c & d
\end{vmatrix}
\]

---

\subsection*{4. Derivatives and Integrals}

To write derivatives, integrals, and partial derivatives, you can use the following syntax:

\textbf{Derivatives:}
\[
\frac{dy}{dx}, \quad \frac{d^2y}{dx^2}
\]

\textbf{Partial Derivatives:}
\[
\frac{\partial y}{\partial x}, \quad \frac{\partial^2 y}{\partial x^2}
\]

\textbf{Integrals:}
\[
\int_0^1 x^2 \, dx, \quad \int_{-\infty}^{\infty} e^{-x^2} \, dx
\]

\textbf{Multiple Integrals:}
\[
\int\!\!\!\int_D f(x,y) \, dx \, dy
\]

---

\subsection*{5. Limits and Summations}

Limits and summations are easy to write in LaTeX:

\textbf{Limits:}
\[
\lim_{x \to 0} \frac{1}{x}, \quad \lim_{n \to \infty} a_n
\]

\textbf{Summation:}
\[
\sum_{n=1}^{\infty} \frac{1}{n^2}
\]

\textbf{Product:}
\[
\prod_{i=1}^{n} x_i
\]

---

\subsection*{6. Piecewise Functions}

Piecewise functions can be written using the \texttt{cases} environment:

\[
f(x) =
\begin{cases}
    x^2 & \text{if } x > 0 \\
    -x & \text{if } x \leq 0
\end{cases}
\]

---

\subsection*{7. Multiline Equations with Numbering}

For equations that span multiple lines and require custom numbering, use the \texttt{align} environment:

\begin{align}
    f(x) &= x^2 + 2x + 1 \tag{1} \\
    g(x) &= \sin(x) + \cos(x) \tag{2} \\
    h(x) &= e^x \tag{3}
\end{align}

You can also use the \texttt{multline} environment when you want to split a long equation over multiple lines, where the first line is left-aligned and the last line is right-aligned:

\begin{multline}
    f(x) = x + x^2 + x^3 + \dots + x^n + \\
    y + y^2 + y^3 + \dots + y^n
\end{multline}

---

\subsection*{8. Custom Operators}

You can define your own custom operators using the \texttt{DeclareMathOperator} command:

\DeclareMathOperator{\foo}{Foo}
\[
\foo(x) = x^2 + 1
\]

---

\subsection*{9. Bold Math Symbols with \texttt{bm}}

Use the \texttt{bm} package to create bold mathematical symbols, which is especially useful for vectors and matrices:

\[
\bm{A} \bm{x} = \bm{b}
\]

You can also bold specific symbols like Greek letters:

\[
\bm{\alpha}, \quad \bm{\beta}
\]

---

\subsection*{10. Fractions and Continued Fractions}

You can write fractions and continued fractions easily in LaTeX:

\textbf{Regular Fraction:}
\[
\frac{a}{b}
\]

\textbf{Continued Fraction:}
\[
a_0 + \cfrac{1}{a_1 + \cfrac{1}{a_2 + \cfrac{1}{a_3 + \ddots}}}
\]

---

\subsection*{11. Spacing in Math Mode}

To adjust the spacing between elements in math mode, you can use the following commands:

\[
a \, b \quad a \; b \quad a \quad b \quad a \qquad b
\]

- \texttt{\textbackslash,} for thin space.
- \texttt{\textbackslash;} for medium space.
- \texttt{\textbackslash quad} for wide space.
- \texttt{\textbackslash qquad} for extra wide space.

---

\subsection*{12. The \texttt{cases} Environment}

The \texttt{cases} environment can also be used for writing systems of equations:

\[
\begin{cases}
    x + y = 10 \\
    x - y = 5
\end{cases}
\]


\end{document}
